\newacronym{ac:aosp}{AOSP}{\gls{gls:aosp}}
\newglossaryentry{gls:aosp}
{ name=Android Open Source Project,
	description={Android is an open source operating system for mobile devices and a corresponding open source project led by Google. This site and the Android Open Source Project (AOSP) repository offer the information and source code needed to create custom variants of the Android OS, port devices and accessories to the Android platform, and ensure devices meet the compatibility requirements that keep the Android ecosystem a healthy and stable environment for millions of users \citep{web:aosp}}}
\newglossaryentry{gls:carbon}
{ name=CarbonROM,
	description={CarbonROM is an aftermarket firmware based on the \gls{gls:aosp} created with the purpose of adding versatility and customization to stock Android. Stability is the highest priority. Their vision is to be the best alternative to a stock operating system on an users device \citep{web:carbon}}}
\newglossaryentry{gls:lineage}
{ name=LineageOS,
	description={A free and open-source operating system for various devices, based on the Android mobile platform. The biggest of it's kind with almost 1.8 million active users as of January 2019 \citep{web:lineage}}}
\newglossaryentry{gls:omnirom}
{ name=OmniROM,
	description={OmniROM is our Android custom ROM variant, feature-packed but always with stability as number one priority in mind.
Based on the \gls{gls:aosp} (\acrshort{ac:aosp}) and enriched by their developers with lots of custom enhancements, OmniROM has set out to give their users a great Android experience on their mobile \citep{web:omnirom}}}
\newglossaryentry{gls:timekeep}
{ name=timekeep,
	description={TimeKeep is a small utility to keep track of time and date since the RTC driver on Qualcomm chipset is read-only. It originates from \acrshort{ac:sodp} \citep{web:sodp}}}
\newglossaryentry{gls:z3}
{ name=Sony Xperia Z3,
	description={A Smartphone released in 09/2014 by Sony Mobile.
	Featuring a Qualcomm Snapdragon S801 and 3GB of RAM along with a Full HD screen it was considered a flagship device upon release and is still more than powerful enough to handle modern Software \citep{web:z3}}}
\newacronym{ac:z3}{Xperia Z3}{\gls{gls:z3}}
\newglossaryentry{gls:s9}
{ name=Samsung Galaxy S9,
	description={A Smartphone released in 02/2018 by Samsung Mobile.
	Featuring a Samsung Exynos 9810 Octa Core CPU and 4GB of RAM along with a 1440x2960 screen it is considered a modern day flagship device and is the newest device available from Samsung. Usually a prime example of fast Software updates \citep{web:s9}}}
\newacronym{ac:selinux}{SELinux}{\gls{gls:selinux}}
\newglossaryentry{gls:selinux}
{ name=Security-Enhanced Linux,
	description={Security-Enhanced Linux (SELinux) is a Linux kernel security module that provides a mechanism for supporting access control security policies, including mandatory access controls (MAC) \citep{web:selinux}.}}
\newacronym{ac:hal}{HAL}{\gls{gls:hal}}
\newglossaryentry{gls:hal}
{ name=Hardware Abstraction Layer,
	description={The Hardware Abstraction Layer is a software subsystem for UNIX-like operating systems providing hardware abstraction. Hardware abstractions are sets of routines in software that emulate some platform-specific details, giving programs direct access to the hardware resources \citep{web:HALWikipedia}}}
\newglossaryentry{gls:wpas}
{ name=wpa\_supplicant,
	description={wpa\_supplicant is a cross-platform supplicant with support for WEP, WPA and WPA2 (IEEE 802.11i). It is suitable for desktops, laptops and embedded systems. It is the IEEE 802.1X/WPA component that is used in the client stations. It implements key negotiation with a WPA authenticator and it controls the roaming and IEEE 802.11 authentication/association of the wireless driver \citep{web:ArchlinuxWikiWPASupplicant}}}
\newacronym{ac:sodp}{SODP}{\gls{gls:sodp}}
\newglossaryentry{gls:sodp}
{ name=Sony Open Device Program,
	description={The Sony Open Device Program is an Sony introduced Program where they provide binaries and device sources as well as changes needed to build the \gls{gls:aosp} for their devices \citep{web:sodp}}}
